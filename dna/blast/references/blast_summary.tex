\documentclass[12pt]{book}

\usepackage{tablefootnote}[2011/11/26]% v1.0e

\makeatletter
\newcommand{\spewnotes}{%
\tfn@tablefootnoteprintout%
\global\let\tfn@tablefootnoteprintout\relax%
\gdef\tfn@fnt{0}%
}
\makeatother

\usepackage{hyperref}

\usepackage{graphicx}
\usepackage{array}
\usepackage{import}
\usepackage{algpseudocode}
\usepackage{algorithm}
\usepackage{multirow}
\usepackage{amsmath}
\usepackage{amssymb}

\usepackage{mathtools}
\DeclarePairedDelimiter{\ceil}{\lceil}{\rceil}
\DeclarePairedDelimiter{\floor}{\lfloor}{\rfloor}
\newcommand{\pmf}{\DOTSB\mbox{pmf}}

\usepackage[titles]{tocloft}
\makeatletter
\newcommand*{\tocwithouttitle}{\@starttoc{toc}}
\makeatother

\usepackage{sectsty}
\partfont{\Large\uppercase}

\usepackage{listings}
\usepackage{fontawesome}
% \defcaptionname{english}{\lstlistingname}{Algorithm}

\usepackage{color}

\definecolor{dkgreen}{rgb}{0,0.6,0}
\definecolor{gray}{rgb}{0.5,0.5,0.5}
\definecolor{mauve}{rgb}{0.58,0,0.82}

\lstset{frame=tb,
  language=Java,
  aboveskip=3mm,
  belowskip=3mm,
  showstringspaces=false,
  columns=flexible,
  basicstyle={\small\ttfamily},
  numbers=none,
  numberstyle=\tiny\color{gray},
  keywordstyle=\color{blue},
  commentstyle=\color{dkgreen},
  stringstyle=\color{mauve},
  breaklines=true,
  breakatwhitespace=true
  tabsize=3
}

\usepackage{setspace}

\sloppy

\begin{document}

\noindent BLAST nucleotide database search summary\newline
\noindent Oliver Serang\newline

\section{The nucleotide search problem}
Given our query nucleotide sequence, $A$, and a database of
potentially many candidate sequences to match, $B_1, B_2, \ldots B_n$,
we would like to find the high-scoring Smith-Waterman alignment
matches $(a_1,b_1), (a_2,b_2), \ldots$. That is, we want to find
sequences of $A$ and any $B_i$ that would produce a high score when
aligned with Smith-Waterman. We could do this directly be aligning all
pairs: $(A,B_1), (A,B_2), \ldots (A,B_n)$; however, this would be far
too slow or memory intensive. 

\section{BLAST fundamentals}
\subsection{Splitting into all $k$-mers}
To start with, we use words to find candidate ``seed'' matches. For
some $k$ fixed in advance (\emph{e.g.}, $k=11$), BLAST finds every
$k$-mer found in $A$ and every $k$-mer found in each $B_i$. These
$k$-mers are 

\subsection{Finding high-scoring $k$-mer word matches}
Now, BLAST finds all possible $k$-mer words that match a candidate
``seed'' and produce score of at least some threshold
$score(A_{(j,j+k)},B_{i,(\ell,\ell+k)}) \geq \tau$. This score
function is an ungapped Smith-Waterman, where we allow matches and
mismatches, but do not allow gaps; however, for small $k$-mer words
like these chosen, we can enumerate all $4^k$ possible $k$-mer words
and score them in $O(k\cdot 4^k)$ steps. For $k=11$, this would take
around 44 million steps, and can be done fairly efficiently fast by
using a table of length $4^k$.

\subsection{Finding and extending $k$-mer matches in the database $B_1, B_2, \ldots B_n$}
At this point, we now have the high-scoring candidate $k$-mer words
for each $k$-mer from $A$. For each $k$-mer $q$ in $A$, its
high-scoring candidate $k$-mer words can be searched for exact matches
in the database in every $B_i$. Locations of exact matches $m$ in some
$B_i$ will be marked on the database sequence as a high-scoring
(\emph{i.e.}, $\geq \tau$) ungapped alignment with $q$. These ungapped
alignments can be extended to the left by looking at the next
characters to the left of $q$ in $A$ and the next characters to the
left of $m$ in $B_i$ and extended to the right by looking at the next
characters to the right of $q$ in $A$ and the next characters to the
right of $m$ in $B_i$. The extension will continue to either side
until the score decreases. In this manner, we know that the scores of
the each alignment must be $\geq \tau$, because they started $\geq
\tau$ and never decreased.

\subsection{A ``two-hit'' model}
At this point, we have locally maximal ungapped alignments (between
$A$ and each $B_i$) and scores. These are called ``high-scoring
segment pairs'' (abbreviated HSPs-- not ``heat shock proteins''). HSPs
that are within a certain distance (a constant decided in advance) of
one another will be merged using a local, gapped alignment. These
merged HSPs each consitute a ``hit'', and we will require two such
nearby hits to continue. {\bf Note that this two-hit model is an
  advancement over the original BLAST algorithm; we could assume for
  simplicity that we are only using a ``one-hit'' model where no
  gapped alignment is performed.}

\subsection{Scoring HSP matches}
We now want to convert the alignment scores to something more
probabilistic. Let $s$ be an observed score of an HSP in our data set
and let $S$ be a random variable for the HSP score on a problem such
as this one (\emph{i.e.}, with these query and database sequence
lengths, \emph{etc.}). We would like to compute the chances $\Pr(S\geq
s)$.

\subsection{The Gumbel ``extreme value distribution''}
When selecting the maximum from a large number of roughly independent
trials, the distribution of the maxima converge to a distribution
called the ``extreme value distribution'' (EVD). This is reminscent of
how many added independent random trials may converge to a normal
(\emph{i.e.}, Gaussian) distribution. A Gumbel EVD is one way to
characterize the EVD. It has free parameters $\beta>0$ and $\mu$, and
its cumulative distribution is
\[
\Pr(S\geq s) = 1-e^{-e^{-\frac{s-\mu}{\beta}}}.
\]

Ungapped Smith-Waterman alignments between unrelated strings (a
reasonable null hypothesis, since we want to eliminate the case where
the two strings $A,B_i$ have no matches except those found by chance)
follow EVD, because we have many roughly independent match scores as
we slide $A$ over $B_i$; therefore, under the null hypothesis, these
ungapped alignment scores should have the score distribution
above. The free parameters $\mu$ and $\beta$ can be fit on the HSP
scores that we observe. This will not require searching iteratively;
we search once, fit the parameters, and then report the probability
scores, and revealing those that are most significant.

Note that this EVD convergence has only been proven for ungapped
alignments; the bold font above slightly discusses how we can do the
filtering/search process with gapped or ungapped alignments; to be
perfectly precise, we should use ungapped alignments for this;
however, the gapped ``two-hit'' model, seems to work quite well in
practice.

\subsection{A note on repetitive regions}
Repetitive regions, \emph{e.g.}, ``GGGGGGGGGGGGGG\ldots'' or
``GTAGTAGTAGTAGTAGTA\ldots'' may produce \emph{many} $k$-mer matches;
as such, it can be useful to filter them out in advance. This may
substantially reduce the number of matching $k$-mer regions in any
pair $(A,B_i)$.

\end{document}


