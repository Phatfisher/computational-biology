\documentclass[12pt]{book}

\usepackage{tablefootnote}[2011/11/26]% v1.0e

\makeatletter
\newcommand{\spewnotes}{%
\tfn@tablefootnoteprintout%
\global\let\tfn@tablefootnoteprintout\relax%
\gdef\tfn@fnt{0}%
}
\makeatother

\usepackage{hyperref}

\usepackage{graphicx}
\usepackage{array}
\usepackage{import}
\usepackage{algpseudocode}
\usepackage{algorithm}
\usepackage{multirow}
\usepackage{amsmath}
\usepackage{amssymb}

\usepackage{mathtools}
\DeclarePairedDelimiter{\ceil}{\lceil}{\rceil}
\DeclarePairedDelimiter{\floor}{\lfloor}{\rfloor}
\newcommand{\pmf}{\DOTSB\mbox{pmf}}

\usepackage[titles]{tocloft}
\makeatletter
\newcommand*{\tocwithouttitle}{\@starttoc{toc}}
\makeatother

\usepackage{sectsty}
\partfont{\Large\uppercase}

\usepackage{listings}
\usepackage{fontawesome}
% \defcaptionname{english}{\lstlistingname}{Algorithm}

\usepackage{color}

\definecolor{dkgreen}{rgb}{0,0.6,0}
\definecolor{gray}{rgb}{0.5,0.5,0.5}
\definecolor{mauve}{rgb}{0.58,0,0.82}

\lstset{frame=tb,
  language=Java,
  aboveskip=3mm,
  belowskip=3mm,
  showstringspaces=false,
  columns=flexible,
  basicstyle={\small\ttfamily},
  numbers=none,
  numberstyle=\tiny\color{gray},
  keywordstyle=\color{blue},
  commentstyle=\color{dkgreen},
  stringstyle=\color{mauve},
  breaklines=true,
  breakatwhitespace=true
  tabsize=3
}

\usepackage{setspace}

\sloppy

\begin{document}

\noindent Felsenstein's pruning algorithm for computing the likelihood of a phylogeny\newline
\noindent Oliver Serang

\section{Notation}

\begin{tabular}{r|l}
  $T$ & The phylogenetic tree (including both its topology and its branch lengths)\\
  $D = (D_1, D_2, \ldots)$ & All known (\emph{i.e.}, observed) nucleotide sequences\\
  $X = (X_1, X_2, \ldots)$ & All unknown ancestor nucleotide sequences in the tree\\
  $D_1$ & A known (\emph{i.e.}, observed) nucleotide sequence\\
  $X_1$ & An unknown ancestor nucleotide sequence\\
  $X_{1,2}$ & The second base pair of an unknown ancestor nucleotide sequence\\
  $A \rightarrow B$ & Sequence $A$ becomes sequence $B$\\
  $c(X_1)$ & The set of children of $X_1$ in the tree\\
  $d(X_1)$ & The set of descendents of $X_1$ in the tree\\
  $\downarrow X_1 = d(X_1) \bigcap D$ & All observed descendents of $X_1$ in the tree
\end{tabular}

\section{Direct likelihood computation}

Denote the root as $X_1$.

\begin{eqnarray*}
  \Pr(D | T) &=& \sum_x \Pr(D, X=x | T)\\
  &=& \sum_{x_1} \Pr(\downarrow X_1, X_1=x_1 | T)\\
  &=& \sum_{x_1} \Pr(X_1=x_1) \cdot \Pr(\downarrow X_1 | X_1=x_1, T)
\end{eqnarray*}
The runtime of this naive approach will have cost $\in \Omega({4^k}^n)$
where $k$ is the number of base pairs and $n$ is the number of
species.

\[ \Pr(\downarrow X_i | X_i=x_i, T) = \prod_{Y \in c(X_i)\mbox{ in }T} \sum_y \Pr(x_i \rightarrow y | T) \cdot \Pr(\downarrow Y | Y=y, T) \]

Thus, $\Pr(\downarrow X_i | X_i=x_i, T)$ can be decomposed into
$\Pr(\downarrow Y | Y=y, T)$ for each $Y \in c(X_i)$.

Each term $\sum_y \Pr(x_i \rightarrow y | T) \cdot \Pr(\downarrow Y |
Y=y, T)$ can be cached: $\forall x_i,~ g(x_i) = \sum_y f(x_i,
y)$. These results can be passed up the tree as $g(x_i)$ in a memoized
manner.

The runtime of this ``pruned'' approach will have cost $\in O(4^k \cdot n)$
where $k$ is the number of base pairs and $n$ is the number of
species.


\begin{figure*}
  \centering
  \includegraphics[width=5in]{phylogeny.pdf}
  \caption{{Example phylogenetic tree on 6 species with 5 unknown ancestor species.} 
  \label{fig:phylogeny}}
\end{figure*}

\section{Using an independent nucleotide model}

\[ \Pr(X_1=x_1) = \prod_j \Pr(X_{1,j} = x_{1,j} \]
\[ \Pr(A \rightarrow B | T) = \prod_j \Pr(A_j \rightarrow B_j | T) \]

Given the tree\footnote{This is important: a mutation at one base can
  imply a greater evolutionary divergence, which can imply a greater
  mutation probability at another base; however, once we know $T$,
  what happens at one base will only rarely influence what happens at
  another base.}, each base pair functions independently; thus, we
could essentially draw a separate, independent copy of our phylogeny
for each base pair. Then, we can compute $\Pr(D | T) = \Pr(D_1, D_2,
\ldots | T) = \prod_j \Pr(D_{1,j}, D_{2,j}, \ldots |T)$.

By applying the pruning method above to each tree, we achieve an algorithm that runs $\in O(k \cdot n)$. This can be run on very large data sets.

\end{document}


